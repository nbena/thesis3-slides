\begin{frame}
    \frametitle{MoonCloud}
    \begin{itemize}
        \item Framework per la \alert{valutazione}
        ed il \alert{monitoraggio continuo} di servizi
        cloud
        % \item valutazione continuata che
        % \alert{proprietà} siano rispettate nel servizio cloud
        % (\textit{non solo proprietà di sicurezza})
        \item valutazione che certe proprietà (non solo di sicurezza)
        siano rispettate
        mediante \alert{raccolta continua di evidenze}
        \item per l'utente finale MoonCloud è offerto \alert{as-a-service}
        \begin{itemize}
            \item inserisce informazioni sul target
            \item MoonCloud effettua valutazione
            \item mostra risultati
        \end{itemize}
    \end{itemize}
\end{frame}

\begin{frame}
    \frametitle{Non solo cloud}
    % \begin{itemize}
    %     \item MoonCloud offre un approccio \alert{generico} per
    %     valutare qualsiasi tipo di proprietà in un sistema cloud
    %     \item estendere MoonCloud per analizzare infrastrutture IT
    %     \textit{classiche}
    % \end{itemize}
    \alert{Obiettivo} della tesi: estendere MoonCloud per poter analizzare
    reti aziendali classiche o cloud private

        \begin{itemize}
            \item cercando di rimanere ancora \alert{as-a-service} 
            \item la rete target è protetta da almeno un firewall/NAT, le sue
            risorse non sono accessibili dall'esterno
            \item soluzione: utilizzare una \alert{VPN} tra MoonCloud e le reti dei
            clienti
        \end{itemize}
\end{frame}

\begin{frame}
    \frametitle{Soluzione}
    La VPN deve:
    \begin{itemize}
        \item essere \alert{flessibile}
        \item \alert{ligthweight} per il cliente: non deve configurare niente
    \end{itemize}

    Soluzione:
    \begin{itemize}
        \item device \alert{Linux} portato nella rete target che fa da VPN client
        \item in MoonCloud i \alert{VPN server}
        \item \alert{OpenVPN} per il collegamento VPN 
        \item \alert{nftables} (successore
        di \alert{iptables}) per risolvere problemi di configurazione
    \end{itemize}
\end{frame}


%section sfide principali
\begin{frame}
    \frametitle{Sfida 1 -- NAT al contrario}

    \begin{itemize}
        \item IP src dei pacchetti MoonCloud verso la rete target
        appartiene alla rete MoonCloud
        \item la rete target deve inviare le risposte al VPN client, ma senza
        rotte configurate le invierebbe al proprio default gateway
    \end{itemize}

    \alert{NAT al contrario}: tutti i pacchetti provenienti dalla VPN vengono
    immessi nella rete target usando come IP sorgente quello del client VPN
    \begin{itemize}
        \item stesso NET ID della rete target
        \item quindi le risposte possono tornargli senza problemi
        \item realizzato con \alert{nftables}
    \end{itemize}
\end{frame}

\begin{frame}
    \frametitle{Sfida 2 -- IP mapping}
    ``\textit{Ogni rete connessa alla VPN deve stare in reti IP diverse}''\footnote{\url{https://openvpn.net}}
    \begin{itemize}
        \item si vuole che un server gestisca il maggior numero di reti target diverse
        \item alta probabilità che due reti abbiano lo stesso NET ID
    \end{itemize}

    \alert{IP mapping}: \textit{mappare} ogni rete target in una nuova rete
    \alert{garantita univoca} perché scelta da MoonCloud
    \begin{itemize}
        \item tutta MoonCloud conosce solo indirizzi mappati quindi unici
    \end{itemize}
\end{frame}

\begin{frame}
    \frametitle{Sfida 2 -- IP mapping (2)}
    \begin{enumerate}
        \item Quando si registra un nuovo cliente, le sue reti vengono \alert{mappate}
        in reti nuove ed univoche

        \item il cliente specifica il target dell'analisi usando l'indirizzo IP reale
        
        \item MoonCloud ne ottiene la \alert{versione mappata} in maniera tutto
        \alert{trasparente}: è il l'IP dst dell'analisi

        \item l'analisi parte, nel \alert{VPN client}
        \begin{itemize}
            \item richieste MoonCloud $\rightarrow$ host target:
            \begin{enumerate}
                \item modifica IP dst mappato $\rightarrow$ IP originale
                \item applica \textit{NAT al contario} ed invia ai target
            \end{enumerate}
            \item risposte target $\rightarrow$ MoonCloud:
            \begin{enumerate}
                \item applica inverso di \textit{NAT al contrario}
                \item modifica IP src originale $\rightarrow$ IP mappato
            \end{enumerate}
            % \item pacchetti MoonCloud $\rightarrow$ client: 
            % \item 
            % \item pacchetti client $\rightarrow$ MoonCloud: modifica IP src originale $\rightarrow$ IP mappato
        \end{itemize}

        % \item \alert{Lato server}
        % \begin{itemize}
        %     \item quando si registra un nuovo client si \textit{mappano} le sue reti su reti nuove
        %     \item \alert{trasparente} per l'utente
        %     \begin{itemize}
        %         \item inserisce come target l'IP originale
        %         \item internamente MoonCloud ne ottiene la versione mappata e lavora con essa
        %     \end{itemize}
        % \end{itemize}

    \end{enumerate}

    \begin{itemize}
        \item per fare il mapping lato client si usa \alert{nftables}
    \end{itemize}
\end{frame}

\begin{frame}
    \frametitle{Altre sfide}
    \begin{itemize}
        \item \alert{Staticità configurazione server}
        \begin{itemize}
            \item OpenVPN si configura con file di testo letti all'avvio
            \item necessità di specificare per ogni client quali reti raggiunge per inserire
            rotte nel kernel dell'OS
            \item \alert{Soluzione}: sfruttare hook \alert{\texttt{client-connect}} per eseguire
            uno script che aggiunga le rotte all'OS ogni volta che un client si connette
            \begin{itemize}
                \item lo script viene riletto ad ogni nuova connessione
                \item \alert{client-disconnect} per rimuoverle
            \end{itemize}
        \end{itemize}

        \item \alert{Rotte server-side}
        \begin{itemize}
            \item il client deve conoscere le reti presenti \textit{dietro} il server per questioni di
            routing
            \item non è possibile sapere a priori quali siano
            \item \alert{Soluzione}: il server applica NAT sui pacchetti verso la VPN
        \end{itemize}
    \end{itemize}
\end{frame}

\begin{frame}
    \frametitle{Considerazioni di sicurezza}
    La VPN potrebbe essere sfruttata da attaccanti. Si usano le seguenti contromisure:
    \begin{itemize}
        \item \alert{lato server} si usano \alert{regole di firewalling} che consentono
        di passare alle richieste di MoonCloud ed alle sole risposte ad esse
        \begin{itemize}
            \item No richieste provenienti dalla rete target
        \end{itemize}
        \item un \alert{host malevolo} nella rete target dovrebbe passare
        due livelli di NAT e le regole lato server
    \end{itemize}
\end{frame}


\begin{frame}
    \frametitle{MoonCloud\_VPN}
    \alert{Microservizio} integrato in MoonCloud per gestire la soluzione VPN
    \begin{itemize}
        \item creazione file di \alert{configurazione} per \alert{OpenVPN}
        \begin{itemize}
            \item \alert{trasferimento} via SSH ai server
        \end{itemize}
        \item gestione \alert{certificati} di client e server
        \begin{itemize}
            \item creazione
            \item revoca e rinnovo mediante \alert{CRL}
        \end{itemize}
        \item gestione \alert{IP mapping}
        \begin{itemize}
            \item assegnazione nuove reti ai client
            \item creazione file di \alert{configurazione} per \alert{nftables}
            \item dato un IP originale ritornare quello mappato
        \end{itemize}
    \end{itemize}
\end{frame}

% OTHER POSSIBILITIES?

% \begin{frame}
%     \frametitle{MoonCloud\_VPN (2)}
%     Tecnologie:
%     \begin{itemize}
%         \item Python
%         \item \alert{HTTP REST API}
%     \end{itemize}
% \end{frame}