\begin{frame}
    \frametitle{MoonCloud}
    \begin{itemize}
        \item Framework per la \alert{valutazione}
        ed il \alert{monitoraggio continuo} di servizi
        cloud
        \item valutazione continuata che
        \alert{proprietà} siano rispettate nel servizio cloud
        (\textit{non solo proprietà di sicurezza})
        \item per l'utente finale MoonCloud è un \alert{servizio}
        offerto via cloud
        \begin{itemize}
            \item inserisce informazioni sul target
            \item MoonCloud effettua valutazione
            \item mostra risultati
        \end{itemize}
    \end{itemize}
\end{frame}

\begin{frame}
    \frametitle{Non solo cloud}
    % \begin{itemize}
    %     \item MoonCloud offre un approccio \alert{generico} per
    %     valutare qualsiasi tipo di proprietà in un sistema cloud
    %     \item estendere MoonCloud per analizzare infrastrutture IT
    %     \textit{classiche}
    % \end{itemize}
    \alert{Obiettivo}: estendere MoonCloud per poter analizzare
    infrastrutture IP \textit{classiche} (reti aziendali)

        \begin{itemize}
            \item mantenendo intatto il modello \alert{as-a-service} di
            MoonCloud
            \item MoonCloud non può semplicemente fare richieste verso
            una rete target, c'è almeno un firewall
            \item soluzione: utilizzare una \alert{VPN} tra MoonCloud e la rete target
        \end{itemize}
\end{frame}

\begin{frame}
    \frametitle{Soluzione}
    La soluzione VPN deve:
    \begin{itemize}
        \item essere \alert{flessibile}
        \item \alert{ligthweight} per il cliente: non deve configurare niente
        nella propria rete
    \end{itemize}

    Soluzione:
    \begin{itemize}
        \item device \alert{Linux} portato nella rete target che fa da VPN client
        \item in MoonCloud i VPN server
        \item \alert{OpenVPN} per il collegamento VPN 
        \item \alert{nftables} (successore
        di \alert{iptables}) per risolvere problemi di configurazione
    \end{itemize}
\end{frame}


%section sfide principali
\begin{frame}
    \frametitle{Sfida 1 -- NAT al contrario}

    \begin{itemize}
        \item<1-> IP sorgente dei pacchetti MoonCloud $\rightarrow$ rete target
        appartiene alla rete MoonCloud
        \item<2-> la rete target deve inviare le risposte al VPN client, ma senza
        rotte configurate le invierebbe al proprio default gateway
    \end{itemize}

    \alert{NAT al contrario}: tutti i pacchetti provenienti dalla VPN vengono
    immessi nella rete target usando come IP sorgente quello del client VPN
    \begin{itemize}
        \item stesso NET ID della rete target
        \item quindi le risposte tornano direttamente ad esso
        \item realizzato con \alert{nftables}
    \end{itemize}
\end{frame}

\begin{frame}
    \frametitle{Sfida 2 -- IP mapping}
    ``\textit{Ogni rete connessa alla VPN deve stare in reti IP diverse}''\footnote{\url{https://openvpn.net}}
    \begin{itemize}
        \item<1-> si vuole che un server gestisca più client, quindi reti, possibili
        \item<2-> ma reti target hanno IP privati
        \item<3-> quindi ci saranno conflitti
    \end{itemize}

    \alert{IP mapping}: \textit{mappare} ogni rete target in una nuova rete
    \alert{garantita univoca} perché scelta da MoonCloud
    \begin{itemize}
        \item tutta MoonCloud conosce solo indirizzi mappati quindi unici
    \end{itemize}
\end{frame}

\begin{frame}
    \frametitle{Sfida 2 -- IP mapping (2)}
    \begin{itemize}
        \item \alert{Lato client} si utilizza \alert{nftables} sul VPN client
        \begin{itemize}
            \item pacchetti MoonCloud $\rightarrow$ client: modifica IP mappato $\rightarrow$ IP originale
            (e poi invia a target)
            \item pacchetti client $\rightarrow$ MoonCloud: modifica IP originale $\rightarrow$ IP mappato
        \end{itemize}
        \item \alert{Lato server}
        \begin{itemize}
            \item quando si registra un nuovo client si \textit{mappano} le sue reti su reti nuove
            \item \alert{trasparente} per l'utente
            \begin{itemize}
                \item inserisce come target l'IP originale
                \item MoonCloud si occupa di mapparlo
            \end{itemize}
        \end{itemize}
    \end{itemize}
\end{frame}

\begin{frame}
    \frametitle{MoonCloud\_VPN}
    \alert{Microservizio} integrato in MoonCloud per gestire la soluzione VPN
    \begin{itemize}
        \item creare file di \alert{configurazione} per \alert{OpenVPN}
        \begin{itemize}
            \item e \alert{trasferimento} via SSH ai server
        \end{itemize}
        \item gestire i \alert{certificati} di client e server
        \begin{itemize}
            \item creare
            \item revocare mediante \alert{CRL}
        \end{itemize}
        \item gestire \alert{IP mapping}
        \begin{itemize}
            \item file di \alert{configurazione} per \alert{nftables}
            \item assegnare nuove reti ai client
            \item dato un IP originale ritornare quello mappato
        \end{itemize}
    \end{itemize}
\end{frame}

% OTHER POSSIBILITIES?

% \begin{frame}
%     \frametitle{MoonCloud\_VPN (2)}
%     Tecnologie:
%     \begin{itemize}
%         \item Python
%         \item \alert{HTTP REST API}
%     \end{itemize}
% \end{frame}